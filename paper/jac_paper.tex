\documentclass[11pt, oneside]{article}   	% use "amsart" instead of "article" for AMSLaTeX format
\usepackage{geometry}                		% See geometry.pdf to learn the layout options. There are lots.
\geometry{letterpaper}                   		% ... or a4paper or a5paper or ... 
%\geometry{landscape}                		% Activate for for rotated page geometry
%\usepackage[parfill]{parskip}    		% Activate to begin paragraphs with an empty line rather than an indent
\usepackage{graphicx}				% Use pdf, png, jpg, or eps§ with pdflatex; use eps in DVI mode
								% TeX will automatically convert eps --> pdf in pdflatex		
\usepackage{amssymb}
\usepackage{amsmath}
\usepackage{float}
\usepackage{natbib}

\title{Survival of Gentrification / Depreciation in Restaurants}
\author{F. M. Marsh and J. A. Clithero}
%\date{}							% Activate to display a given date or no date

\begin{document}

\maketitle

\section{Abstract} \label{abstract}

\section{Introduction} \label{introduction}

Yelp Dataset Papers:

\citep{alghunaim2015vector,byers2012groupon,cawkwell2015tracking,chepurna2015exploiting,feng2013recommendation,gutierrez2014noise,hajas2014analysis,hu2014your,liu2015mining,mashhadi2012accuracy,quattrone2015there}

Zillow Dataset Papers:

Has previously been combined with the yelp dataset \citet{bonnarlittle}.

Our goal is to use the correlation of two time-series:

1. The monthly median rent, as tracked by Zillow Rental Data.
2. The median restaurant review rating (stars) for each restaurant in a neighborhood.

Zillow rental data can be used to detect appreciating, and depreciating neighborhoods.

As rents rise in a given neighborhood, which types of businesses fare / worse better in the reviews? 
As rents fall in a given neighborhood, which types of business fare / worse better in the reviews?

We hope to present concrete suggestions to restaurant owners to improve the survivability of their businesses in times of strong appreciation / depreciation in the housing market.

\section{Data} \label{Data}

\subsection{Yelp Academic Dataset} \label{Yelp Academic Dataset}

The Yelp Academic Dataset contains five files:

1) \texttt{yelp\_academic\_dataset\_business.json} \\

2) \texttt{yelp\_academic\_dataset\_review.json} \\

3) \texttt{yelp\_academic\_dataset\_user.json} \\

4) \texttt{yelp\_academic\_dataset\_checkin.json} \\

The Yelp Dataset Business file includes 

\subsection{Zillow Public Dataset} \label{Zillow Public Dataset}

The Zillow Public Dataset (hereafter Zillow dataset) contains many files.

Zillow divides homes into geographic ``neighborhoods" with well defined boundaries. The Zillow Home Value Index (ZHVI) is Zillow's best estimate of median home price in a neighborhood. ZHVI is reported on a monthly basis for 6,958 neighborhoods across the US.

Median rental price for studio, one, two, three, four and five or more bedroom apartments are reported for a smaller set of about 300 neighborhoods.

Each Zillow neighborhood has geographic boundaries, defined in an associated ESRI arcGIS shape file.

\section{Methods} \label{Methods}

\subsection{Combination of Datasets} \label{Combination of Datasets}

Each Yelp business is tagged with a geographic (latitude, longitude) coordinate. In this section, we describe how we sort each Yelp business into its appropriate Zillow neighborhood.

\begin{figure}[H]
\begin{center}
\includegraphics[width = \textwidth]{phx_test.png}
\caption{Yelp Businesses (points in blue) and Zillow neighborhood boundaries (lines in red) for the Phoenix, AZ metro area. In \S\ref{Combination of Datasets}, we describe how we sort each Yelp Business into its appropriate Zillow neighborhood.}
\label{default}
\end{center}
\end{figure}

To perform this sorting, we employ a two-step approach. In the first step, we test every Yelp business for inclusion in the bounding box of every Zillow neighborhood. In the second step, we test every Yelp business for polygon inclusion in the neighborhoods which bounding boxes it lies within. We use this two-step approach because the first step can rule out all but two or three of the 

We introduce the concept of the bounding box which we will define as the smallest range of latitudes and longitudes that include the whole neighborhood polygon. We test each Yelp business for inclusion in the set of 6,958 bounding boxes. In Fig. \ref{bbox_example}, we see a randomly selected Yelp business, displayed as a red point. We see that this business is included in the bounding boxes of two Zillow neighborhoods.

\begin{figure}[H]
\begin{center}
\includegraphics[width = \textwidth]{bbox_example.png}
\caption{Example of neighborhood and neighborhood bounding box inclusion method. Yelp business 400 (the red point) is included in the bounding boxes of two Zillow neighborhoods. It is only included in one neighborhood polygon, however.}
\label{bbox_example}
\end{center}
\end{figure}

We then test for point-in-polygon inclusion using an implementation of a ray-casting method in \texttt{Python} \citep{sutherland1974characterization}. For each Yelp business, we only test the Zillow neighborhoods whose bounding boxes it lies within. 

\subsection{Description of Combined Dataset}

\begin{figure}[H]
\begin{center}
\includegraphics[width = \textwidth]{metro_zhvi_box.png}
\caption{}
\label{metro_zhvi_box}
\end{center}
\end{figure}

\begin{figure}[H]
\begin{center}
\includegraphics[width = \textwidth]{metro_zhvisqft_box.png}
\caption{}
\label{metro_zhvisqft_box}
\end{center}
\end{figure}

\begin{figure}[H]
\begin{center}
\includegraphics[width = \textwidth]{metro_stars_box.png}
\caption{}
\label{metro_zhvisqft_box}
\end{center}
\end{figure}

\begin{figure}[H]
\begin{center}
\includegraphics[width = \textwidth]{metro_price_box.png}
\caption{}
\label{metro_zhvisqft_box}
\end{center}
\end{figure}

The combined Zillow and Yelp Dataset contains

\begin{table}[H] \label{metro_summary}
\begin{center}
\begin{tabular}{|l|rrrr|}
\hline
Metro &  median (B/N) &        mean (B/N) &    Total B &  Total N \\
\hline
Phoenix, AZ    &   116.5 &  467.6 &  19,640 &   42 \\
Charlotte, NC    &    43.0 &   66.1 &   4,296 &   65 \\
Las Vegas, NV    &   178.0 &  218.6 &   9,619 &   44 \\
Pittsburgh, PA    &    43.5 &   94.8 &   2,464 &   26 \\
Madison, WI    &    36.0 &   71.4 &   1,071 &   15 \\
\hline
\end{tabular}
\end{center}
\caption{}
\end{table}

\begin{table}[H] \label{city_summary}
\begin{center}
\begin{tabular}{|l|rrrr|} 
\hline
City  &  median (B/N) &         mean (B/N) &    Total B &  Total N \\
\hline
Charlotte       &    43.0 &    66.1 &   4296 &   65 \\
Henderson       &   105.0 &   153.0 &   2907 &   19 \\
Las Vegas       &   241.5 &   242.0 &   5809 &   24 \\
Madison         &    36.0 &    71.4 &   1071 &   15 \\
Mesa            &   491.5 &   509.5 &   3057 &    6 \\
North Las Vegas &   903.0 &   903.0 &    903 &    1 \\
Phoenix         &   625.0 &   763.30 &  11450 &   15 \\
Pittsburgh      &    43.5 &    94.8 &   2464 &   26 \\
Scottsdale      &  2100.0 &  1531.7 &   4595 &    3 \\
Tempe           &    26.5 &    29.9 &    538 &   18 \\
\hline
\end{tabular}
\end{center}
\caption{}
\end{table}


\subsection{Determination of Most Common Yelp Tags} \label{Determination of Most Common Yelp Tags}

Each Yelp business has user-generated tags, that allow other users to determine what genre the business is. For restaurants, common tags are "Mexican", "Chinese", etc.

\subsection{Computation of Price Trend} \label{Computation of Price Trend}

\section{Results} \label{Results}

\subsection{Statics Results} \label{Static Results}

In this section, we compute static results for each neighborhood.

\begin{figure}[H]
\begin{center}
\includegraphics[width = \textwidth]{ZHVI_price_reg.png}
\caption{}
\label{ZHVI_price_reg}
\end{center}
\end{figure}



\bibliographystyle{abbrvnat}
\bibliography{jac_paper.bib}

\end{document}  